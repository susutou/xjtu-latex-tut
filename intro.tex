% !Mode:: "TeX:UTF-8"
% The first line is for editors like, WinEdt, to specify the encoding

\section{前言}
我接触\LaTeX{}大概有一年了,从开始的一个absolute beginner到现在,也算是初窥门径,基本能用\LaTeX{}来写日常的实验报告、数学建模论文
乃至毕业论文,从了解文档的基本结构到现在也能写一点文类和宏包(虽然只是一丁点),在这些过程中,我深刻地体会到了\LaTeX{}的高效、稳定
以及高质量,这是Microsoft Word之类的字处理软件(word processor)
所难以企及的高度\footnote{对于一般的需求来说,Word也可以达到可接受的质量,但一定要学会使用样式。}。

因为自己用着觉得不错,我就很想把\LaTeX{}推广给更多的同学使用。目前看来,\LaTeX{}在中国有一定的使用群体,但是仍然很小众,真正欣赏她的人
更是少之又少。在\LaTeX{}在科技排版界已经成为事实标准的今天,
国内的诸多高校以及科技出版机构仍然抗拒使用\LaTeX{},许多老师更是指明要求使用Word排版,我觉得这是一件很可怕的事情。暂不说Word的排版质量,
首先它是一个昂贵的商业软件,虽然针对高校学生有优惠,但是既然有免费并且更先进,几乎可以称为state of the art的\TeX{}系列,我们为什么要花钱
去赞助evil的Microsoft呢?当然,我知道更多的同学还在使用盗版的Office软件,那更应该把他们解放出来了。

说到这里,许多同学可能要问,“\LaTeX{}容易学吗?”“我并不是学工科的,也能用\LaTeX{}吗?”。我在这里想说的是,如果你是计算机科学、数学、物理
及相关专业的,我极力推荐你使用\LaTeX{};如果你是Art类专业的,比如英语专业,需要使用西文完成论文的,我也非常推荐你使用\LaTeX{}——因为\TeX{}
的西文排版引擎是非常先进的,大型专业排版软件如InDesign、PageMaker等也只能勉强与其比肩。

说了这么多,\LaTeX{}与Word这类字处理软件到底有什么本质的区别或优点呢?网上这类对比非常多,就我个人而言,感受到的主要有以下几点:
\begin{enumerate}

\item Word的传统文件格式是私有的、封闭的二进制文件\footnote{自Word 2003起开放了文档格式,为使用zip压缩的一组XML文件,但是也非常复杂,难以解析。},
容易损坏(就不用我问有多少同学的文档被Word弄坏过吧);
而\LaTeX{}的源文件则是是字节码,不容易损坏,并且有很高的可移植性,几乎可以在任何平台打开、编辑,并且可以方便地输出为PDF格式。

\item 相比于Word这一类字处理软件,\LaTeX{}拥有更好的输出效果。在\TeX{}排版引擎内部有极佳的断字以及断行算法,以求达到最佳的美感(仅对于英文来说);
并且,\TeX{}内部对长度的定义非常精细,精细到你无法想象,从而可以对文本进行精确地定位,完成高质量的输出。

\item \LaTeX{}的思想是内容与结构分离,只要预定义好了文档结构,写作者就可以主要关注文档的内容,而不需要边写作还要边考虑文档的格式,比如字体大小、
样式等等与内容无关的内容;而Word虽然是一种“所见即所得”的工具,但是对于格式的实时关注对于写作来说是一个distraction,会分散写作者的注意力。

\item \LaTeX{}对于写作长文档,尤其是科技论文有很大的优势。使用\LaTeX{},章节编号、图形编号、表格编号、参考文献等等都可以自动生成,并且可以方便地
实现交叉引用;而在Word里,类似的功能不仅操作复杂,有时候还要借助第三方的、同样是收费的插件。

\item \LaTeX{}对于数学公式的排版是公认的最好,没有之一。只需要简单地对比一下Word和\LaTeX{}生成的公式效果就可以发现。

\end{enumerate}

除了这些优点,下面我还想针对计算机专业的同学说几句,那就是用\LaTeX{}能更好地与计算机\textbf{交流}。在没有计算机排版的年代,作者要
在自己的底稿上注明文章的结构,由排字工进行排版;而\LaTeX{}继承了这一优良的传统。\LaTeX{}是一个强大的宏(macro)语言,在文档结构上也可以称为
标记语言,常见的标记语言有HTML,XML等。所谓标记语言也就是用一定的标记标明文档的结构,比如你要告诉\LaTeX{}某行字是文章的标题,或者某行字是
一节的小标题,那么应该怎么做呢?Okay,就是像这样:
\begin{quote}
\begin{verbatim}
\title{这是文章标题}
\section{这是节标题}
\subsection{这是小节标题}
\end{verbatim}
\end{quote}
这里,用~\verb+\somecommand{}+~标明文档的结构。

那么,我们再来看一下,在Word里,一般的同学会怎么做:输入一行字,比如“实验目的”,然后选中,
设置为宋体,三号字,加粗,无缩进;如果有新的标题,比如“实验结果”,那么重复一遍上面的过程,或者使用格式刷。嗯,看起来还不错,也许对于鼠标
操作快的同学,甚至还比输入一个标记来得快捷。但是,考虑这样的情况,对于一个长文档,比如毕业论文,某一天老师突然让你把所有的标题改成黑体,那对于
像刚才那样Word操作习惯的同学,必须一个个地改格式;如果老师还希望标题可以编号,那就更惨了。

对于\LaTeX{},这些都可以通过简单的设置来搞定,因为
事先已经用标记把文章的结构标好了,只需要对指定的结构使用指定的格式,就可以一次性地把问题解决;并且,\LaTeX{}可以自动为章节编号,并且可以按你
指定的格式来编号。当然,Word也有样式功能和大纲模式,这与\LaTeX{}的工作方式类似,但是据我了解,没有合适的引导,大部分同学是不会用样式和大纲模式的,
并且,Word也没有直观地提示你使用。况且,用符号(或语言)与计算机交流的效率和准确性远远高于使用鼠标指点。(所以,计算机系的同学们,动手coding吧,不要
怕写程序,要学会真正与计算机交流的方式。)

总而言之,在我看来,对于计算机专业的同学来说,\LaTeX{}是再合适不过的一个文档准备软件了,更不必提她是基于\TeX{}的,而\TeX{}是计算机科学界的巨擘
Prof.~Donald E. Knuth的大作。同时,\LaTeX{}也是向大部分科技类国际期刊、会议投稿的第一选择;如果你用心,也照样可以用\LaTeX{}作出优美的艺术品,
当然,\TeX{}本身就是一件艺术品。
